%%%%%%%%%%%%%%%%%%%%%%%%%%%%%%%%%%%%%%%%%%%%%%%%%%%%%%%%%%%%%%%%%%%%%
%%%
%%% Resumo
%%%
%%%%%%%%%%%%%%%%%%%%%%%%%%%%%%%%%%%%%%%%%%%%%%%%%%%%%%%%%%%%%%%%%%%%%

\chapter*{Resumo}
\addcontentsline{toc}{chapter}{Resumo}

\noindent
A aplica��o de modelos computacionais baseados em modelagem compartimental na epidemiologia � amplamente estudada, como pode-se observar na extensa literatura dispon�vel. A simula��o de din�micas epidemiol�gicas � de particular interesse no estudo, preven��o e controle de doen�as. Com base nestas premissas, este trabalho busca abordar o problema de simula��o de hipot�ticas doen�as baseadas em modelo compartimental tipo SEIRS fazendo uso de t�cnicas de sistemas multiagentes, modelagem de indiv�duos sob a especifica��o de palavras bin�rias, como em t�cnicas de \textit{bitstring}, uso de dados georreferenciados para a especifica��o e composi��o de um \textit{lattice} apropriado � simula��o e a paraleliza��o do sistema de simula��o utilizando GPGPU na plataforma \textit{CUDA}. Como resultado pr�tico pretende-se apresentar uma aplica��o totalmente funcional, capaz de simular eventos epidemiol�gicos computacionais em uma regi�o da cidade de Cascavel. A avalia��o do modelo proposto ser� executada atrav�s da realiza��o de experimentos num�rico-computacionais, buscando compar�-los com aqueles obtidos da literatura e investigando ainda aspectos computacionais e perform�ticos da implementa��o realizada.

\vspace{1cm}
\noindent
\textbf{Palavras-chave:} Epidemiologia, Sistemas multiagentes, Modelos compartimentais, modelagem \textit{bitstring}, plataforma computacional paralela \textit{CUDA}


