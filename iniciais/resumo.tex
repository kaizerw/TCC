%%%%%%%%%%%%%%%%%%%%%%%%%%%%%%%%%%%%%%%%%%%%%%%%%%%%%%%%%%%%%%%%%%%%%
%%%
%%% Resumo
%%%
%%%%%%%%%%%%%%%%%%%%%%%%%%%%%%%%%%%%%%%%%%%%%%%%%%%%%%%%%%%%%%%%%%%%%

\chapter*{Resumo}
\addcontentsline{toc}{chapter}{Resumo}

\noindent
A aplica��o de modelagem compartimental na epidemiologia � amplamente estudada, como pode-se observar na extensa literatura dispon�vel. A simula��o de din�micas epidemiol�gicas � de particular interesse no estudo, preven��o e controle de doen�as transmiss�veis. Com base nestas premissas, este trabalho busca abordar o problema de simula��o de hipot�ticas doen�as baseadas em modelagem compartimental SEIRS, por meio de sistema multiagente. Para obter uma adequada formula��o � simula��o em GPGPU na plataforma CUDA, ao modelo � empregada a metodologia de \textit{bitstring}. Ao ambiente computacional s�o utilizadas t�cnicas e software especificamente desenvolvido para manipular dados georreferenciados � especifica��o e composi��o de um \textit{lattice} apropriado � simula��o computacional. Como resultado pretende-se apresentar uma aplica��o funcional e adequada para simular eventos epidemiol�gicos em uma regi�o da cidade de Cascavel/PR. A avalia��o do modelo proposto � executada atrav�s da realiza��o de experimentos num�rico-computacionais, onde os resultados s�o comparados aos obtidos na literatura, sobretudo nos aspectos quantitativos �s implementa��es realizadas.

\vspace{1cm}
\noindent
\textbf{Palavras-chave:} Epidemiologia computacional, Sistema multiagente, Modelos compartimentais, Modelagem \textit{bitstring}, Plataforma Computacional CUDA.


