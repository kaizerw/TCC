% Universidade Estadual do Oeste do Paran� - UNIOESTE
% Centro de Ci�ncias Exatas e Tecnol�gicas
% Curso de Bacharelado em Inform�tica
% Arquivo: Aprova.Tex
% Conte�do: Folha de Aprova��o para o TCC.

%%%%%%%%%%%%%%%%%%%%%%%%%%%%%%%%%%%%%%%%%%%%%%%%%%%%%%%%%%%%%%%%%%%%%
%%%
%%% folha de aprova��o
%%%
%%%%%%%%%%%%%%%%%%%%%%%%%%%%%%%%%%%%%%%%%%%%%%%%%%%%%%%%%%%%%%%%%%%%%

\begin{center}
\fontsize{12}{12}
\textbf{WESLEY LUCIANO KAIZER}\\
\vspace{3cm}
\fontsize{14}{14}
\textbf{UM MODELO MULTIAGENTE \textit{BITSTRING} EM CUDA PARA SIMULAR A PROPAGA��O DE HIPOT�TICAS DOEN�AS BASEADAS EM MODELAGEM COMPARTIMENTAL TIPO SEIRS}\\
\vspace{3cm}
\fontsize{10}{10}
Monografia apresentada como requisito parcial para obten��o do T�tulo de Bacharel em Ci�ncia da Computa��o, pela Universidade Estadual do Oeste do Paran�, Campus de Cascavel, aprovada pela Comiss�o formada pelos professores:\\
\vspace{2cm}
\begin{flushright}
\begin{minipage}[10cm] {8.5cm}
\begin{center}
\rule{6cm}{0.01mm}\\
Prof. Dr. Rog�rio Lu�s Rizzi (Orientador)\\
Colegiado de Matem�tica, UNIOESTE\\
\vspace{1cm}
\rule{6cm}{0.01mm}\\
Profa. Dra. Claudia Brandelero Rizzi\\
Colegiado de Ci�ncia da Computa��o, UNIOESTE\\
\vspace{1cm}
\rule{6cm}{0.01mm}\\
Prof. Dr. Guilherme Galante\\
Colegiado de Ci�ncia da Computa��o, UNIOESTE\\
\end{center}
\end{minipage}
\end{flushright}
\vspace{3.3cm}
Cascavel, \today
\end{center} 